% This is "sig-alternate.tex" V2.0 May 2012
% ---------------------------------------------------------------------------------------------------------------
% This .tex source is an example which *does* use
% the .bib file (from which the .bbl file % is produced).
% REMEMBER HOWEVER: After having produced the .bbl file,
% and prior to final submission, you *NEED* to 'insert'
% your .bbl file into your source .tex file so as to provide
% ONE 'self-contained' source file.
%

\documentclass{sig-alternate}
\usepackage{epstopdf}
\DeclareGraphicsExtensions{.png}
\graphicspath{{img/}}

\begin{document}
%
% --- Author Metadata here ---
\conferenceinfo{CHARLOTTESVILLE}{'15 Charlottesville, Virginia USA}
%\CopyrightYear{2007} % Allows default copyright year (20XX) to be over-ridden - IF NEED BE.
%\crdata{0-12345-67-8/90/01}  % Allows default copyright data (0-89791-88-6/97/05) to be over-ridden - IF NEED BE.
% --- End of Author Metadata ---

\title{Math Retrieval Using Leaf-Root Expression Trees
Format}

\numberofauthors{2} %  in this sample file, there are a *total*
% of EIGHT authors. SIX appear on the 'first-page' (for formatting
% reasons) and the remaining two appear in the \additionalauthors section.
%
\author{
% You can go ahead and credit any number of authors here,
% e.g. one 'row of three' or two rows (consisting of one row of three
% and a second row of one, two or three).
%
% The command \alignauthor (no curly braces needed) should
% precede each author name, affiliation/snail-mail address and
% e-mail address. Additionally, tag each line of
% affiliation/address with \affaddr, and tag the
% e-mail address with \email.
%
% 1st. author
\alignauthor
Andrew Norton\titlenote{}\\
       \affaddr{University of Virginia}\\
       \affaddr{Charlottesville, Virginia}\\
       \email{apn4za@virginia.edu}
% 2nd. author
\alignauthor
Ben Haines\titlenote{}\\
       \affaddr{University of Virginia}\\
       \affaddr{Charlottesville, Virginia}\\
       \email{bmh5wx@virginia.edu}
}
% There's nothing stopping you putting the seventh, eighth, etc.
% author on the opening page (as the 'third row') but we ask,
% for aesthetic reasons that you place these 'additional authors'
% in the \additional authors block, viz.
% Just remember to make sure that the TOTAL number of authors
% is the number that will appear on the first page PLUS the
% number that will appear in the \additionalauthors section.

\maketitle
\begin{abstract}
This paper describes our implementation of a system for
searching for mathematical expressions. The paper summarize
existing techniques for math retrieval and describes our particular
implementation. We propose a few extensions to implement and 
provide experimental results that measure the effectiveness of
these proposals.
\end{abstract}

% A category with the (minimum) three required fields
%\category{H.4}{Information Systems Applications}{Miscellaneous}
%A category including the fourth, optional field follows...
%\category{D.2.8}{Software Engineering}{Metrics}[complexity measures, performance measures]

%\terms{Theory}

%\keywords{Information Retrieval, path, text tagging}

\section{Introduction}
\subsection{Background}
Within the field of Information Retrieval, the task of mathematical
expression retreival is a topic that has been attracting an increasing
amount of attention in recent years. Math search is useful in a variety
of situations, particularly for students and practitioners of technical
fields. Particular scenarios include researchers who want to discover
relevant work relating to a particular function or a student who needs
help solving a particular problem. Existing solutions are often
unsatisfying. For example, consider arxiv.org and math.stackexchange.com.
These sites are two of the largest resources of collected mathematical
information for both professional researchers and student. However, the
usefulness of much of this information is decreased by challenges in
locating it. The search feature of the arXiv does not permit searching for
commonplace symbols such as "+" or "-". Searches on stackexchange often
return no relevant results despite additional efforts revealing that multiple
relevant results do exist. A better math retrieval system could in both cases
increase productivity of many users.

Multiple approaches to parsing, indexing, and searching mathematical
expression have been proposed. A specific state of the art algorithm
does not exist as research in the field has not had time to converge
to optimal solutions for the problems of storing and parsing expressions.
Approaches tend to fall into one of two categories, those that use
established text based search methods with modifications in order
to apply them to the particular problem of Mathematical Information
Retrieval (MIR) and those that use tailored approaches that attempt
to take advantage of the inherent structure of expressions to improve
performance. We provide a brief comprison of techniques in these 
categories and justify our particular choice of a system to implement. 
Our research focuses on the effectiveness of the Leaf-Root path system
for representing structured expressions. The primary contribution of the
work is the suggestion of query expansion in order to allow searches for
generalized expressions and an examination of how changes in the 
parsing grammar can effect performance.

\section{Related Work}
Among the earliest discussions of a math retrieval system, and
one of the few describing an actual large scale implementation
is the paper by Youssef and Miller regarding the Digital Library
of Mathematical Functions\cite{youssef:library}. The idea proposed involves
a sequence of steps to process mathematical notation into a format
recognizable by existing search engines.
This involves first using macros to map math symbols to standard alphanumeric text representations.
For example "$+$" and "$<$" are mapped respectively to "plus" and "lt".
Next, nested expressions, such as exponents, are flattened. Finally,
expressions are normalized by sorting the leaves of the corresponding
parse tree in a standardized manner. 

The authors select an evolutionary approach that augments 
existing text search engines due to practicality constraints
but they also outline a few relevant challenges that they suggest
could be better addressed with a structural approach. These challenges
include 
\begin{itemize}
	\item Recognition of mathematical symbols
	\item Capturing and indexing structure
	\item Accounting for mathematical "synonyms"
\end{itemize}

Since the 2003 publishing of the paper discussed above, incremental
improvements have been suggested by a variety of sources. In 2007 Miner
and Munavalli introduced a more involved for processing inputs while
still relying on a standard text search infrastructure for the 
fundamental search operation.\cite{miner:approach} Largely within the last five years a number of 
papers have emerged that attempt to address the problems in more
fundamental ways.
In 2012 an approach that bridges the gap between text and structure based
methods is introduced in the paper "A structure based
approach for mathematical expression retrieval".\cite{kumar:structure} The fundamental
idea is to use a modified longest common substring (LCS) algorithm to 
measure similarity between expressions. The expressions are tokenized
and each token is given an label to indicate its nested depth. The LCS
algorithm then weights the similarities between two expressions by how
closely depth markers are aligned. 
An alternate structure capturing approach that has been explored in
a number of papers is that of substitution trees.\cite{kolhase:mathwebsearch}\cite{yuan:layout} In such a tree each
internal node represents a generalized expression form and leaves
represent specific expressions for which all variables have a 
substituted value. A simple example tree is shown in the next figure.
\begin{figure}
	\centering
	\includegraphics{subtree}
	\caption{A sample substitution tree}
\end{figure}
This structure stores relationships between abstract expression forms.
Within each node individual expressions are stored as symbol layout trees
that map spatial relationships between individual components of an expression.
The authors of this approach reported performance improvements when compared
to a standard Lucene search that they attribute to substructure queries 
enabled by the tree structures. 

The approach that most directly influenced our implementation is an
alternative method for using trees to store structure. Furthering our
intuition that a parse trees could be a useful representation of structure
the particular implementation by Zhong provided inspiration and 
solutions to some technical challenges.\cite{zhong:cowpie}

\section{Proposed Solution}
There are three general components to our proposed solution.
Each will be discussed separately and the particular 
decisions we made will be justified. 

\subsection{Representation}
The system was inspired by and particularly targets the Stack Exchange
and arXiv use cases which both store math information as LaTeX. 
Additionally, the corpus of expressions made available to us contains
\LaTeX expressions. For these reasons we chose to focus on searching
for \LaTeX exressions rather than alternatives such as MathML. As a 
result of this we decided to accept \LaTeX formatted inputs. The advantage
of this choice is that \LaTeX is a widely used system for expressing
mathematical structures, in particular, users who want to search
a \LaTeX corpus are probably also capable of composing queries with it.
As well as this, many simple expressions can be given without any 
overt formatting and will still be valid which increases ease of use.

When given a raw string, whether it be as part of the corpus or in 
a query the system first performs a number of preprocessing operations.
First, operations such as "\textbackslash left and \textbackslash right"
purely change the display of an expression and not the structure or
content are removed. Second, semantically similar operations are grouped roughly into equivalence classes and occurrences are replaced with a single symbol. For example, "\textbackslash pmod, \textbackslash bmod, and \textbackslash mod" which have the same meaning but different 
presentations are replaced with the common command "MOD". Other
operations that might have clear differences in meaning but are also
similar in some way such as "+, \textbackslash sum, and \textbackslash bigoplus" are replaced with the common command "SUM".


Because the entire article is contained in
the \textbf{document} environment, you can indicate the
start of a new paragraph with a blank line in your
input file; that is why this sentence forms a separate paragraph.

\subsection{Citations}
Citations to articles \cite{bowman:reasoning,
clark:pct, braams:babel, herlihy:methodology},
conference proceedings \cite{clark:pct} or
books \cite{salas:calculus, Lamport:LaTeX} listed
in the Bibliography section of your
article will occur throughout the text of your article.
You should use BibTeX to automatically produce this bibliography;
you simply need to insert one of several citation commands with
a key of the item cited in the proper location in
the \texttt{.tex} file \cite{Lamport:LaTeX}.
The key is a short reference you invent to uniquely
identify each work; in this sample document, the key is
the first author's surname and a
word from the title.  This identifying key is included
with each item in the \texttt{.bib} file for your article.

The details of the construction of the \texttt{.bib} file
are beyond the scope of this sample document, but more
information can be found in the \textit{Author's Guide},
and exhaustive details in the \textit{\LaTeX\ User's
Guide}\cite{Lamport:LaTeX}.

This article shows only the plainest form
of the citation command, using \texttt{{\char'134}cite}.
This is what is stipulated in the SIGS style specifications.
No other citation format is endorsed or supported.

\section{Conclusions}
This paragraph will end the body of this sample document.
Remember that you might still have Acknowledgments or
Appendices; brief samples of these
follow.  There is still the Bibliography to deal with; and
we will make a disclaimer about that here: with the exception
of the reference to the \LaTeX\ book, the citations in
this paper are to articles which have nothing to
do with the present subject and are used as
examples only.
%\end{document}  % This is where a 'short' article might terminate

%ACKNOWLEDGMENTS are optional
\section{Acknowledgments}
This section is optional; it is a location for you
to acknowledge grants, funding, editing assistance and
what have you.  In the present case, for example, the
authors would like to thank Gerald Murray of ACM for
his help in codifying this \textit{Author's Guide}
and the \textbf{.cls} and \textbf{.tex} files that it describes.

%
% The following two commands are all you need in the
% initial runs of your .tex file to
% produce the bibliography for the citations in your paper.
\bibliographystyle{abbrv}
\bibliography{sigproc}  % sigproc.bib is the name of the Bibliography in this case
% That's all folks!
\end{document}
